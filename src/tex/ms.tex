% Define document class
\documentclass[twocolumn]{aastex631}
\usepackage{showyourwork}
\usepackage[version=4]{mhchem}

% Begin!
\begin{document}

% Title
\title{Assessing Phonon Properties of Shape Memory Alloys using Graph Neural Network Potentials}

% Author list
\author{@stefanbringuier}

% Abstract with filler text
\begin{abstract}

\end{abstract}

% Main body with filler text
\section{Introduction}
\label{sec:intro}


\section{Results}
\label{sec:results}
\subsection{Equation of States}
\label{subsec:eos}

The thermodynamic equation of state (EOS) is a standard assement of the phase stability of materials at a give condition. In the context of atomic calculations the equation of state can be determined at ground-state (i.e., 0K) to asses the phase stability of the different crystals. The equation of state can be represented as a pressure vs. volume or more commonly as the energy per atom  vs the ratio of the isotropically strained volume to that of the equilibrium volume. One criteria for a interatomic potential is that it accurately refelcts the EOS curves for different phases. This entails the construction of a convex hull which represents the minimum energy curve accross the structures for the volume strain conditions. \par

The EOS curves for the \ce{NiTi} structures \textit{B2}, \textit{B19}, \textit{B19'}, \textit{B33}, and \textit{Pbcm}  for the different potentials studied in this work are shown in fig.~\ref{fig:eos}
\begin{figure}[ht!]
    \script{PlotNiTiEOS.py}
    \begin{centering}
        \includegraphics[width=\linewidth]{figures/NiTi_EOS_Comparison.png}
        \caption{
            The equation of state curves for different \ce{NiTi} structures for (a) Mutter (EAM) (b) Zhong (EAM) (c) M3GNet (d) CHGNet (e) MACE and (f) ALIGNN. 
        }
        \label{fig:eos}
    \end{centering}
\end{figure}


\appendix
\section{Reproducibilty}
This study was carried out using the reproducibility software \href{https://github.com/showyourwork/showyourwork}{\showyourwork} \citep{Luger2021}, which leverages continuous integration to programmatically download the data from \href{https://zenodo.org/}{zenodo.org}, create the figures, and compile the manuscript. Each figure caption contains two links: one to the dataset stored on zenodo used in the corresponding figure, and the other to the script used to make the figure (at the commit corresponding to the current build of the manuscript). The git repository associated to this study is publicly available at \url{https://github.com/stefanbringuier/SMA_Phonons_GNNIP} and allows anyone to re-build the entire manuscript. The datasets generated by this paper are stored at \url{xyz}.

\bibliography{bib}

\end{document}
