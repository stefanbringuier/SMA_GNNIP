% Define document class
\documentclass[preprint]{elsarticle}
\usepackage{showyourwork}

%% Packages
\usepackage{orcidlink}
\usepackage[utf8]{inputenc}
\usepackage{fontspec}
\usepackage{fontawesome5}
\usepackage{amssymb}
\usepackage{amsmath}
\usepackage{siunitx}
\usepackage[version=4]{mhchem}
\usepackage{pdflscape}
%% Handles multipage table. Not sure allowed by pub.
\usepackage{longtable}
\usepackage{subcaption}
\usepackage{listings}
\lstset{
  basicstyle=\ttfamily,
  columns=fullflexible,
  frame=single,
  breaklines=true,
  postbreak=\mbox{\textcolor{red}{$\hookrightarrow$}\space},
}

% \journal{Computational Materials Science}

\begin{document}


\title{Assessing Phonon Properties of Shape Memory Alloys using Graph Neural Network Potentials}

% Author list
\author[1]{Stefan Bringuier\corref{cor1}\orcidlink{0000-0001-6753-1437}}
\ead{stefanbringuier@gmail.com}
\cortext[cor1]{Corresponding author}
\affiliation[1]{organization={Indpendent Researcher},
  city={San Diego, CA},
  country={United States},
}

% Abstract with filler text
\begin{abstract}
Shape Memory Alloys (SMAs) such as NiTi are pivotal in vibrational dampening,robotic arms, prothestic hands, and mechanical vales as a result of their unique displacive phase transformation properties. However, assessing their phonon characteristics, which provide insight into the displacive phase transformation, by using conventional Density Functional Theory (DFT) methods is computationally challenging. In this paper we assess the use of recent Graph Neural Network (GNN) potentials as a viable alternative. These GNN potentials have leveraged large DFT-computed databases for training and provide a good approximation to the potential energy surfaces which enable inference on energies and forces in lattice and atomic dynamic calculations. We compare GNN potentials with the Embedded Atom Method (EAM) and Modified EAM style potentials in predicting phonon properties of NiTi and PtTi. We find that in general the EAM/MEAM potentials are slightly more performant than the GNN in that they appear to more accurately describe equation of state, phonon dispersions, and elastic constants. However, the MACE potential shows significant promise as an exploritory potential for designing and characterizing more complex SMAs. Our findings suggest GNN potentials as a promising initial avenue for advancing SMA research, in particular ternary and quaternary systems, upon fine-tuning of pre-trained GNN models with more relevant DFT data (i.e., force-constants) to permit better outcomes.   
\end{abstract}

\begin{keyword}
  shape memory alloys \sep machine learning potentials \sep phonons
\end{keyword}


\maketitle

 % Main body with filler text
\section{Introduction}
\label{sec:intro}
Shape memory alloys derive their unique displacive phase transformations from the dynamic instablity that occurs during the application of temperature, strain, or stress. The origin of this instability occurs due to softening or symmetry breaking of the phonon modes in the phase. Therefore, improved understanding of the phonon characteristics of the different phases that can occur for a given SMA is particular important. One of the challenges is in performing such calculations, which typically employ density functional theory in order to determine the atomic forces, which in turn are used to construct the dynamical matrix. However, the use of classical interatomic potentials can alleviate some of this computational burden. \par

Recently, there as been a growing effort in the development of machine learning potentials. These deviate from traditional physics approaches to interatomic potentials, which have historically been inspired from more accurate methods like tight-binding method. In machine learning potentials, one describes the atoms and structure using descriptors of features, these are typically learned or prescribed. The features can be prescribed in isolation or combination of  structural, compositional, and property characteristics of the material data point. Futhermore, one may choose to fined new representations via unsurpervised learning. \par

The subsequent step once the featureization process is completed is to determine a suitable neural network archetcture which can learn the target property or properties given the feature inputs. In the case of atomistic simulations the target is prediction of the potential energy surface and corresponding gradients (i.e., forces). Two of the most early attempts to do this were the works of \citet{Bartok2010} and \citet{Thompson2015}, whereby projection onto higher-dimensional basis, such as bispectrum componenets of local neighbor density, is learned in order to describe the atomic environments. The energy per atom is then determined by these learned features by employing linear regression or non-linear Gaussian process regression. \par

While these early attempts were successfully, they could only include structural details to learn new features from. Futhermore, the training process was typically restricted to a limited number of elemental constieunts and therefore for each chemical system and phases of interest one needs to provide a suitable training dataset. To move beyond such limitations, groups began using graph neural network (GNN) architectures and message passing scheme to represent the atomic structures as nodes and edges \cite{Reiser2022}. This enabled two distinctions, first being that the nodes and edges corresponding to the atoms and bonds could include features, and secondly the learning could be done over a large structural and compositional space enabling a generalization\footnote{Typically these are referred to as Universal interatomic potentials or forcefields since they in theory generalize well to any chemical system or phase.}. The use of GNNs is efficient in representing structure-composition-property inputs to additional NN layers for predicting potential energy surfaces and gradients and was initially pursued in the CGCNN, MEGNet, and ALIGNN potentials \cite{Choudhary2021,Chen2019,Xie2018}. The differences between the approaches are predominately in atomic and bond features, nearest-neighbor graph, and specific NN archetcture. \par

The training of these early GNN utlized large density functional theory calculation repositories and mostly tuned the learning cost function to ground-state structure and energy. This resulted in mean-absolute-errors and other measures of performance that could be improved upon \cite{Riebesell2023}. More recently there have been significant improvements in both training data used and the GNN architecture that show promise for truly approaching a general interatomic potential suitable for static and dynamic atomistic calculations of multicomponents and multiphases. If succesfull, identifying and designing new potential SMA materials using GNN interatomic potentials could have tremendous impact on technology. Therefore, the primary goal is to provide a qualitative assement of predictions of the equations of state, phonon bandstructure, and elastic constants for various phases of NiTi and PtTi using three GNN potentials intended for general use: M3GNet, CHGNet, and MACE \cite{Chen2022,Deng2023,Batatia2022}. We note that the \citet{Riebesell2023} provides a summarized performance over these GNN using the materials project database, but does not specifically address SMA and phonon characteristics. \par   


\section{Computation Methodology}
\label{sec:methods}

\subsection{Workflow}
\label{sec:workflow}
The ground-state structures for the NiTi phases \textit{B2}, \textit{B19}, \textit{B19'}, and \textit{BCO} are generated using the atomic simulation environment (ASE) package \cite{Larsen2017}. This requires that the unit cell parameters, basis atoms, and spacegroup number be provided. The initial unit cell parameters and basis corrdinates for these structures have been taken from the materials project \cite{Jain2013} and several references \cite{Haskins2016,Kadkhodaei2018}. The initial structures for NiTi are shown in fig~\ref{fig:niti_structures} and are adapted for PtTi as well given same spacegroup symmetries. The energy and forces on the atoms are determined using the ASE calculator interfaces for EAM, MEAM, M3GNet, CHGNet, and MACE.\cite{Mutter2010,Zhong2011,Ko2015,Kim2017,Chen2022,Deng2023,Batatia2022} The EAM and MEAM potential interfaces are achieved through the use of LAMMPS python library \cite{Thompson2022} and ASE. The ALIGNN ASE calculator interface was also implemented, however, we found that the predictions for NiTi and PtTi to be unreliable are not included in this study.\par

For each combination of potential/model, chemical system, and phase we perform calcualtions for the structure minimization, equation of state (EOS), phonons, and elastic constants. To determine the finite-temperature phase transitions we utlize an isothermal-isobaric ensemble via molecular dynamics simulations. The thermostating and barostating is done with Nose-Hoover and Parrinello-Rahman dynamics and the tempertaure is adjusted in a quasi-static manner which entails equlibrating the simulation for 10 ns while decrementing the temperature by 10 Kelvin; the pressure is kept constant at 1 bar. We note that in order to achieve the different computing environments required by the GNN potentials we leverage the ability to specify the python computing environments using \href{https://github.com/showyourwork/showyourwork}{\showyourwork} package.\par

\begin{figure}[ht!]
    \script{VisualizedStructures.py}
    \begin{centering}
        \includegraphics[width=\linewidth]{figures/NiTi_VisualizedStructures.png}
        \caption{
          Initial unit cells of used in this work for NiTi, (a) \textit{B2}(221), (b) \textit{B19}(11), (c) \textit{B19'}(51), and (d) \textit{BCO}(61). The spacegroup numbers are given in the parentheses. Also additional atoms (i.e. non-basis sites) are shown based on periodic boundary conditions. For PtTi (a-b) structures are used with different unit cell parameters and basis coordinates. 
        }
        \label{fig:niti_structures}
    \end{centering}
\end{figure}

Prior to any property calculations or dynamics the structures are optimized using the FIRE\cite{Bitzek2006} local optimizer with the forces converged to $2\cdot 10^{-3}$eV/\AA. We observed that other local optimization routines find the same ground-state structures. To ensure that the unit cell parameters were adjusted along with the atomic coordinates and that the crystal symmetry is not broken, we utlize a constraint and filter during the optimization process. \par

\subsubsection{Equation of State}
The EOS can be obtained by varying the cell volume and calculating the total system energy without allowing for atomic relaxations. Starting with the obtained ground-state structures the unit cell volume is adjusted by applying an isotropic strain,

\begin{equation}
  \label{eq:isotropic_strain}
  T=\begin{bmatrix}
    \epsilon & 0 & 0 \\
    0 & \epsilon & 0 \\
    0 & 0 & \epsilon \\
  \end{bmatrix}
\end{equation},

to the cell and scaling the fractional atomic coordinates. Here $\epsilon$ is the fraction strain applied to the unit cell. To ensure that the strain is indeed isotropic we provide a quality check on the symmetry of the distorted structure using the spacegroup package \textit{spglib} \cite{Togo2018}. The isotropic strain is applied in compression and tension, from $-11\%$ to $11\%$, which gave a good range to characterize the EOS curves and obtain analytical fits. We fit the EOS using a third order inverse polynomial,

\begin{equation}
  \label{eq:eos_fit}
  E(V) = c_0 + c_1 t + c_2 t^2 + c_3 t^3, \quad \text{where} \quad t = V^{-\frac{1}{3}}
\end{equation},

where the coefficients are used to determine equilibrium volume, system energy, and bulk modulus \cite{Alchagirov2003} at 0K. In order to compare respective EOS amongst the different phases, we normalize the strained volume of each phase with respect to the obtained minimized ground-state volume. \par

\subsubsection{Phonon Bandstructures}
Phonon calculations are performend using finite-difference method via small displacement of atoms in a reference cell.\cite{Alfe2009} In the small displacement method, atoms in the primitive cell are slightly displaced to calculate forces on other atoms, constructing the force constant matrix, which, when used with the dynamical matrix, allows computing phonon frequencies at any q-vector in the Brillouin zone. This requires typically requires the use of a supercell of the original unit cell. In order to ensure convergence of this method, one needs to sweep different supercell sizes and atomic displacements. This process is done for each interatomic potential and structure. The supercells used in this work  have a repeat in any one direction ranging from $5-8$ and depends on the crystal system. The atomic displacement ranges from $0.03-0.26$\AA\,and was found to strongly depend on interatomic potential and crystal system. Details on the settings are provided in the appendix~\ref{sec:phonon_calc_config}. \par

In order to sample high-symmetry points in the irreducible Brillouin zone (IBZ) for specific structures, we sample from locations specified in the Bilbao crystallographic \textit{KVEC} utility\cite{Aroyo2014}, although the ASE structures object does provide a standardized routine for sampling the irreducible Brillouin zone for the specific structures.\cite{Setyawan2010}. We predominately focus on high-symmetry points in the IBZ that correspond to the corners and along directions known to exhibit phonon instabilities. The IBZ and points symmetry point locations can be seen in appendix \ref{sec:appx_ibz}. The final phonon dispersion diagrams are plot along these points. \par

To better understand dynamic instability in the various structures, the same isotropic strain in eq.~\ref{eq:isotropic_strain} is used and the repsective phonon bandstructures are calculated. These are then plotted to observed the onset or change in stability or instability of phonon different branches. \par

\subsubsection{Elastic Constants}


\subsubsection{Finite-Temperature Phase Transitions}

\subsection{Physics Based Potentials}
\label{sec:physics_potentials}

The EAM and MEAM potentials are 


\subsection{Graph Neural Network Potentials}



\section{Results}
\label{sec:results}

\subsection{NiTi: Ground State Structures}
\label{subsec:nitigs}

%\comment{Display results in tables}

\variable{output/Table_NiTi_Equilibrium_Structures.tex}

\subsection{NiTi: Equation of States}
\label{subsec:nitieos}

The thermodynamic equation of state (EOS) is a standard assement of the phase stability of materials at a give condition. In the context of atomic calculations the equation of state can be determined at ground-state (i.e., 0K) to asses the phase stability of the different crystals. The equation of state can be represented as a pressure vs. volume or more commonly as the energy per atom  vs the ratio of the isotropically strained volume to that of the equilibrium volume. One criteria for a interatomic potential is that it accurately refelcts the EOS curves for different phases. This entails the construction of a convex hull which represents the minimum energy curve accross the structures for the volume strain conditions. \par

The EOS curves for the \ce{NiTi} structures \textit{B2}, \textit{B19}, \textit{B19'}, \textit{B33}, and \textit{Pbcm}  for the different potentials studied in this work are shown in fig.~\ref{fig:eos}. The two EAM potentials, show that at 0 strain ($V/V_o = 1.0$) both the \textit{B19P} and \textit{B33} phases are stable. I believe this is the first reporting of the \textit{B33} phase for these two fittings of the EAM potential for \ce{NiTi}. Its important to note that these potentials have been developed to capture the displacive phase transformation between \textit{B19P} to \textit{B2}. This is shown in fig.~\ref{fig:eos}(a,b) with the stability indicated upon compressive isotropic strain. A mininum appears around $\approx 0.9$. At tensile isotropic strains the \textit{B19P} phase is stable up till around $1.18$ where the \textit{Pbcm} phase shows stability. An interesting note is that these two fittings of the EAM potential show a double well curve for the \textit{B2} phase. \par

In the UIP-GNN potentials M3GNet, CHGNet, and MACE, the equation of states shown in fig.~\ref{fig:eos}(c-e) look fairly different from thos in fig.~\ref{fig:eos}(a,b). For the M3GNet,  at no strain, the \textit{B33} appears to be the predicted stable phase and the \textit{B2} phase becomes stable around $0.9$ (see fig.~\ref{fig:eos}(c). The stability for \textit{B19} and \textit{B19P} exit, in trade-off, between approximately $1.025$ and $1.1$ tensile isotropic strain. Interestingly beyond a strain of $1.1$ the \textit{Pbcm} is dominate. The double-well curve shown for the EAM potentials is not present for the B2-phase. The CHGNet potential results shown in fig.~\ref{fig:eos}(d) are similar to the M3GNet, but the stability window of the \textit{B2} occurs at a lower compressive isotropic strain. \par

The MACE potential shown in fig.~\ref{fig:eos}(e) has a very different set of EOS curves compard to the M3GNet and CHGNet. Here many of the phases exhibit a stablity window, except the \textit{B33} phase, which was stable in M3GNet and CHGNet. Interestingly, the \text{B19} phase, not the \text{B19P} phase is shown to be the stable zero strain phase. One unique perspective of the the UIP-GNN models is that they provide a different displacive phase transformation pathway then that shown by the EAM potentials. However, this does not include the meta-stable R-phase or other potential meta-stable phases which can also offer such transformation mechanisms. \par


Finally, the ALIGNN potential EOS was generated, but a few issues arose during the calculations. The most prominant was the difficulty in convergence of the structural optimzation to the specified tolerance (see \ref{sec:methods}. Secondly the calculated cohesive energies differs from the UIP-GNN by a factor $\approx 2$ and by $1$ eV for the EAM potentials. The \textit{B19P} phase is the stable at zero strain and for a considerable portion of the compressive, but all of the tensile. At compressive strain beyond $\approx 0.94$ the \textit{B2} phase is stable. 

\begin{figure}[ht!]
    \script{PlotCohesiveEnergy.py}
    \begin{centering}
        \includegraphics[width=\linewidth]{figures/NiTi_CohesiveEnergyPlot.png}
        \caption{
          Comparison of predicted cohesive energy for respective ground-state structure and model.
        }
        \label{fig:ecoh}
    \end{centering}
\end{figure}


\begin{figure}[ht!]
    \script{PlotEOS.py}
    \begin{centering}
        \includegraphics[width=\linewidth]{figures/NiTi_EquationOfStates.png}
        \caption{
            The equation of state curves for different \ce{NiTi} structures for (a) Mutter (EAM) (b) Zhong (EAM) (c) Ko (2NN-MEAM) (d) M3GNet (e) CHGNet (f) MACE interatomic potentials. The x-axis is the ratio of the isotropic strained unit cell volume, $V$,  with respect to the ground-state unit cell volume, $V_o$.
        }
        \label{fig:eos}
    \end{centering}
\end{figure}

\subsection{NiTi: Phonon Instability}
\label{subsec:niphonons}


\subsubsection{B2 Phase}
\label{subsubsec:b2}

\begin{figure}[!htp]
    \script{PlotPhonons.py}
    \begin{centering}
        \includegraphics[width=\linewidth]{figures/NiTi_B2_ModelsPhononBandstructures.png}
        \caption{
          Ground-state, no isotropic strain, Phonon bandstructure of B2 phase for all models. The Mutter and Zhong EAM potentials differ in that Mutter predicts no dynamic instability of the B2 phase at 0K. 
        }
        \label{fig:allmodels_b2}
    \end{centering}
\end{figure}

\begin{figure}[!htp]
    \script{PlotPhonons.py}
    \begin{centering}
        \includegraphics[width=\linewidth]{figures/NiTi_B19_ModelsPhononBandstructures.png}
        \caption{
          Ground-state phonon bandstructure of B19 phase for all models.
        }
        \label{fig:allmodels_B19P}
    \end{centering}
\end{figure}

\begin{figure}[!htp]
    \script{PlotPhonons.py}
    \begin{centering}
        \includegraphics[width=\linewidth]{figures/NiTi_B19P_ModelsPhononBandstructures.png}
        \caption{
          Ground-state phonon bandstructure of \textit{B19}$^\prime$ phase for all models.
        }
        \label{fig:allmodels_B19P}
    \end{centering}
\end{figure}


\begin{figure}[!htp]
    \script{PlotPhonons.py}
    \begin{centering}
        \includegraphics[width=\linewidth]{figures/NiTi_BCO_ModelsPhononBandstructures.png}
        \caption{
          Ground-state phonon bandstructure of BCO phase for all models.
        }
        \label{fig:allmodels_B19P}
    \end{centering}
\end{figure}

\begin{figure}[!htp]
    \script{PlotPhonons.py}
    \begin{centering}
        \includegraphics[width=0.5\linewidth]{figures/NiTi_Mutter_B2_StrainsPhononBandstructures.png}
        \vspace{1mm}
        \includegraphics[width=0.5\linewidth]{figures/NiTi_Zhong_B2_StrainsPhononBandstructures.png}
        \vspace{1mm}
        \includegraphics[width=0.5\linewidth]{figures/NiTi_Ko_B2_StrainsPhononBandstructures.png}
        \caption{
           Phonon dispersion of B2 structure for the  (top) EAM Mutter, (middle) EAM Zhong, and (bottom) 2NN-MEAM Ko potentials at different isotropic strains. The optical branches have been made more transparent to emphasis the behavior in the TA$_1$ and TA$_2$ branches. The color indicates direction from compressive to tensile strain. Note that the strain percent shown is with respect to isotropic changes (i.e. volume). 
        }
        \label{fig:mutter_zhong_phonon_b2}
    \end{centering}
\end{figure}


%ADD Plot of zero strain EAM and first-pricinpals results

\begin{figure}[!htp]
    \script{PlotPhonons.py}
    \begin{centering}
      \includegraphics[width=0.5\linewidth]{figures/NiTi_M3GNet_B2_StrainsPhononBandstructures.png}
      \vspace{1mm}
      \includegraphics[width=0.5\linewidth]{figures/NiTi_CHGNet_B2_StrainsPhononBandstructures.png}
      \vspace{1mm}
      \includegraphics[width=0.5\linewidth]{figures/NiTi_MACE_B2_StrainsPhononBandstructures.png}
      \caption{
        Phonon dispersion of B2 structure for the (top) M3GNet, (middle) CHGNet, and (bottom) MACE potentials for different isotropic strains. Similar transcparency is used as done in \ref{fig:mutter_zhong_phonon_b2}.
      }
      \label{fig:gnn_phonon_b2}
    \end{centering}
\end{figure}


The B2 phase of NiTi exhibits notable phonon instabilities, particularly at the \textit{M} point in the Brillouin zone. These instabilities are characterized by imaginary modes, represented as negative phonon frequencies, indicating a dynamical instability in this crystal structure. The \textit{M} point, located at the edge of the Brillouin zone, corresponds to a wave vector of $\left[\frac{1}{2}, \frac{1}{2}, 0\right]$ in units of the reciprocal lattice vectors. The phonon dispersion curves for NiTi in the B2 phase reveal a softening of the acoustic modes at the \textit{M} point, suggesting a propensity for a structural phase transition. \par

The mode Gruneisen parameter, $\gamma$, plays a pivotal role in understanding the stabilization of the NiTi B2 phase under external strain, particularly in the context of phonon instabilities at the M point. The mode Gruneisen parameter is defined as,

\begin{equation}
  \label{eq:modegruneisen}
  \gamma = -\frac{V}{\omega}\left(\frac{\partial \omega}{\partial V}\right)
\end{equation},

where $V$ is the volume, usually taken at equilibrium, and $\omega$ is the phonon frequency. For the B2 phase of NiTi, the negative phonon frequencies at the M point indicate an instability that can be modulated by applying strain. The strain perturbs the lattice volume, thereby altering the phonon frequencies. The effect of this strain can be quantitatively assessed using the mode Gruneisen parameter, which describes how the phonon frequencies change with volume. A positive $\gamma$ suggests that the phonon frequency increases as the volume decreases under strain, potentially stabilizing the lattice. In the case of NiTi, applying compressive strain may increase the frequencies of the soft modes at the M point, thus stabilizing the B2 phase and delaying or suppressing the martensitic transformation. This interplay between strain and phonon behavior, quantified by $\gamma$, is crucial in tailoring the mechanical properties of shape memory and superelastic materials.


\variable{output/Table_NiTi_M_ModeGruneisen.tex}



\subsection{PtTi: Equation of States}

\section{Conclusion}
\label{sec:conclusion}
This word aimed to provide an assement of the of GNN univeral interatomic potentials on the ability to capture the stable ground-state phases of binary and ternary SMA. In addition, the dynamic stability of these SMA is crucially evaluated to understand the suitability in observing the onset of displacive phase transformations via the softening of TA branch modes. In the case of NiTi we find that all the GNN potentials predict either the Monoclinic or BCO phases as stable over the B2 structure. However, not all agree with the recent works by (cite) which have indicated that it is the BCO phase which is the ground-state structure.

We observe that the M3GNet and CHGnet capture the M-point instability of the B2 phase, however, no stablizing isotropic strain is observed. With regard to the MACE, very good agrement is found with the phonon bandstructure and that from ab-initio methods. The ALIGNN potential was found not to be suitable of NiTi.


\bibliographystyle{elsarticle-num-names}
\bibliography{references}


\section*{CR\lowercase{e}d\lowercase{i}T author statement}

S.B: Conceptualization, Formal Analysis, Investigation, Methodology, Software, Visualisation, Writing---original draft.


\section*{Reproducibilty and Data Availability}
This study was carried out using the reproducibility software \href{https://github.com/showyourwork/showyourwork}{\showyourwork}, which leverages continuous integration to programmatically download the data from \href{https://zenodo.org/}{zenodo.org}, create the figures, and compile the manuscript. Each figure caption contains two links: one to the dataset stored on zenodo used in the corresponding figure, and the other to the script used to make the figure (at the commit corresponding to the current build of the manuscript). The git repository associated to this study is publicly available at \url{https://github.com/stefanbringuier/SMA_Phonons_GNNIP} and allows anyone to re-build the entire manuscript. The datasets generated by this paper are stored at \url{xyz}.

\section*{Acknowledgements}

\appendix
\subsection{Phonon Calculation Configurations}
\label{sec:phonon_calc_config}

The code used to generate the phonon configuration settings for each potential mode and structure are shown the table below. Settings are used for both NiTi and PtTi systems.

\variable{output/Phonons_ConfigSettingsCode.tex}

\subsection{Irreducible Brillouin Zones}
\label{sec:appx_ibz}

The irreducible Brillouin zones for the the different crystals investigated in this study are provided below. The high-symmetry points are indicated. Additionally the coordinates of the high-symmetry points in the IBZ are shown in the tables below. In Ref. xxx the  specific implementation of the band path sampling and definitions are provided.

\begin{figure}[!htp]                                                                                      
    \script{AppendixBZ.py}                                                                           
    \begin{centering}                                                                                    
      \includegraphics[width=0.5\linewidth]{figures/B2_BrillouinZonePointsSampled.png} 
      \vspace{1mm}                                                                                       
      \includegraphics[width=0.5\linewidth]{figures/B19_BrillouinZonePointsSampled.png}
      \vspace{1mm}
      \includegraphics[width=0.5\linewidth]{figures/B19P_BrillouinZonePointsSampled.png}
      \vspace{1mm}
      \includegraphics[width=0.5\linewidth]{figures/BCO_BrillouinZonePointsSampled.png}
      \vspace{1mm}                                                                                       
      \caption{The irreducible Brillouin zones for \textit{B2}(\#221), \textit{B19}(\#51), \textit{B19'}(\#11), and \textit{BCO}(\#63) crystal structures}                                                               \label{fig:ibz}                                                                          
    \end{centering}                                                                                      
\end{figure} 

\variable{output/B2_SpecialSymmetryPointsBZ.tex}
\variable{output/B19_SpecialSymmetryPointsBZ.tex}
\variable{output/B19P_SpecialSymmetryPointsBZ.tex}
\variable{output/BCO_SpecialSymmetryPointsBZ.tex}


\end{document}
