% Define document class
\documentclass[preprint,colorlinks=true,linkcolor=black,citecolor=black]{elsarticle}
\usepackage{showyourwork}

%% Packages
\usepackage{orcidlink} \usepackage[utf8]{inputenc}
\usepackage{fontspec} \usepackage{fontawesome5} \usepackage{amssymb}
\usepackage{amsmath} \usepackage{siunitx}
\usepackage[version=4]{mhchem} \usepackage{pdflscape}
%% Handles multipage table. Not sure allowed by pub.
\usepackage{longtable} \usepackage{tabularx} \usepackage{subcaption}
\usepackage{listings} \lstset{ basicstyle=\ttfamily,
  columns=fullflexible, frame=single, breaklines=true,
  postbreak=\mbox{\textcolor{red}{$\hookrightarrow$}\space}, }

% \journal{Computational Materials Science}

\begin{document}

\title{Qualitative Assessesment of Graph Neural Network Potentials for
	Phonon Properties of Shape Memory Alloys using Graph Neural Network
	Potentials}

% Author list
\author[1]{Stefan
	Bringuier\corref{cor1}\orcidlink{0000-0001-6753-1437}}
\ead{stefanbringuier@gmail.com} \cortext[cor1]{Corresponding author}
\affiliation[1]{organization={Indpendent Researcher}, city={San Diego,
			CA}, country={United States}, }

% Abstract with filler text
\begin{abstract}
	Shape Memory Alloys (SMAs) such as NiTi are pivotal in vibrational
	dampening,robotic arms, prothestic hands, and mechanical vales as a
	result of their unique displacive phase transformation
	properties. However, assessing their phonon characteristics, which
	provide insight into the displacive phase transformation, by using
	conventional Density Functional Theory (DFT) methods is
	computationally challenging. In this paper we assess the use of
	recent Graph Neural Network (GNN) potentials as a viable
	alternative. These GNN potentials have leveraged large DFT-computed
	databases for training and provide a good approximation to the
	potential energy surfaces which enable inference on energies and
	forces in lattice and atomic dynamic calculations. We compare GNN
	potentials with the Embedded Atom Method (EAM) and Modified EAM
	style potentials in predicting phonon properties of NiTi and
	PtTi. We find that in general the EAM/MEAM potentials are slightly
	more performant than the GNN in that they appear to more accurately
	describe equation of state, phonon dispersions, and elastic
	constants. However, the MACE potential shows significant promise as
	an exploritory potential for designing and characterizing more
	complex SMAs. Our findings suggest GNN potentials as a promising
	initial avenue for advancing SMA research, in particular ternary and
	quaternary systems, upon fine-tuning of pre-trained GNN models with
	more relevant DFT data (i.e., force-constants) to permit better
	outcomes.
\end{abstract}

\begin{keyword}
	shape memory alloys \sep machine learning potentials \sep phonons
\end{keyword}

\maketitle

% Main body with filler text
\section{Introduction}
\label{sec:intro}

The design and development of shape memory alloys (SMA) via
\textit{in-silico} methods would provide great value in advancing
structural, medical, and other technologies. In order to enable such
activity would require sampling from very large structural and
compositional spaces and evaluating material properties indicative of
displacive phase transformation phenomena. The use of density
functional theory calculations to obtain properties such as equations
of state, phonon disperisons, and elastic constants, can be done,
however, such calculations are computationally demanding and are
potentially prohibitive for smaller groups trying to develop new
SMAs. To alleviate such restriction, one can resort to classical
interatomic potentials which can greatly improve computational cost in
evaluating the energies and forces more. However, the immediate
challenge is to whether a suitable interatomic potential exist given a
chemical system, and moreover, how transferable is it to different
compositions and phases. \par

Recently, there as been a growing effort in the development of machine
learning potentials. These deviate from traditional physics approaches
to interatomic potentials, which have historically been inspired from
more accurate methods like tight-binding method. In machine learning
potentials, one describes the atoms and structure using descriptors of
features, these are typically learned or prescribed. The features can
be prescribed in isolation or combination of structural,
compositional, and property characteristics of the material data
point. Futhermore, one may choose to fined new representations via
unsurpervised learning. \par

The subsequent step once the featureization process is completed is to
determine a suitable neural network (NN) architecture which can learn
the target property or properties given the feature inputs. In the
case of atomistic simulations the target is prediction of the
potential energy surface and corresponding gradients (i.e.,
forces). Two of the most early attempts to do this were the works of
\citet{Bartok2010} and \citet{Thompson2015}, whereby projection onto
higher-dimensional basis, such as bispectrum components of local
neighbor density, is learned in order to describe the atomic
environments. The energy per atom is then determined by these learned
features by employing linear regression or non-linear Gaussian process
regression. We note that these are just two of the early examples but
similar potentials have been developed in earlier, in parallel, or
recently by various groups
\cite{Behler2007,Wang2018,Deringer2019,Lee2019}. \par

While these early attempts using ML models were successfully, they may
only include limited representations of structural and/or chemical
details to learn new features from which could limit
transferability. Futhermore, the training process was typically
restricted to a limited number of elemental constieunts of the target
material system. As a consequence, for each chemical system and phases
of interest one needs to provide a suitable training dataset. To move
beyond such limitations, groups began using graph neural network (GNN)
architectures and message passing scheme to represent the atomic
structures as nodes and edges \cite{Reiser2022}. This enabled two
distinctions, first being that the nodes and edges corresponding to
the atoms and bonds could include features, and secondly the learning
could be done over a large structural and compositional space enabling
a generalization\footnote{Typically these are referred to as Universal
	interatomic potentials or forcefields since they in theory
	generalize well to any chemical system or phase.}. The use of GNNs
is efficient in representing structure-composition-property inputs to
additional NN layers for predicting potential energy surfaces and
gradients and was initially pursued in the CGCNN, MEGNet, and ALIGNN
potentials \cite{Choudhary2021,Chen2019,Xie2018}. The differences
between the approaches are predominately in atomic and bond features,
nearest-neighbor graph, and specific NN archetcture. \par

The training of these early GNN utlized large density functional
theory calculation repositories and mostly tuned the learning cost
function to ground-state structure and energy. This resulted in
mean-absolute-errors and other measures of performance that could be
improved upon \cite{Riebesell2023}. More recently there have been
significant improvements in both training data used and the GNN
architecture that show promise for truly approaching a general
interatomic potential suitable for static and dynamic atomistic
calculations of multicomponents and multiphases. If succesfull,
identifying and designing new potential SMA materials using GNN
interatomic potentials could have tremendous impact on
technology. Therefore, the primary goal is to provide a qualitative
assessment of pre-trained GNN interatomic potentials that are promoted
as being transferable across compositions and phases. We specfically
focus on the predictions for equations of state, phonon bandstructure,
and elastic constants for various phases of NiTi and PtTi using three
GNN potentials: M3GNet, CHGNet, and MACE
\cite{Chen2022,Deng2023,Batatia2022}. We note that the
\citet{Riebesell2023} provides a summarized performance over these GNN
using the materials project database, but does not specifically
address SMA and phonon characteristics. \par

\section{Computation Methodology}
\label{sec:methods}

\subsection{Workflow}
\label{sec:workflow}
To promote reproducible research and align with findable, accessible,
interoper-able and reusable (FAIR) principles\cite{Walsh2024}, we
utilize \showyourwork python workflow\cite{Luger2021} and several open
source python packages. Instructions for reproducing calculated data,
figures, tables, and manuscript are provided in
appendix~\ref{sec:appx_reproduce} and corresponding github
repo\footnote{\url{https://github.com/stefanbringuier/SMA_Phonons_GNNIP}}. We
note that in order to achieve the different computing environments
required by the interatomic potentials we leverage the ability to
specify the python computing environments using
\href{https://github.com/showyourwork/showyourwork}{\showyourwork}
package.\par

\subsection{Calculation Details}
\label{sec:calc_details}
The ground-state structures for the NiTi phases \textit{B2},
\textit{B19}, \textit{B19'}, and \textit{BCO} are generated using the
atomic simulation environment (ASE) package \cite{Larsen2017}. This
requires that the unit cell parameters, basis atoms, and spacegroup
number be provided. The initial unit cell parameters and basis
corrdinates for these structures have been taken from the materials
project \cite{Jain2013} and several references
\cite{Haskins2016,Kadkhodaei2018}. The initial structures for NiTi are
shown in fig~\ref{fig:niti_structures} and are adapted for PtTi as
well given same space group symmetries. The energy and forces on the
atoms are determined using the ASE calculator interfaces for EAM,
MEAM, M3GNet, CHGNet, and
MACE.\cite{Mutter2010,Zhong2011,Ko2015,Kim2017,Chen2022,Deng2023,Batatia2022}
The EAM and MEAM potential interfaces are achieved through the use of
LAMMPS python library \cite{Thompson2022} and ASE. We have also
implemented calculators for potentials by \cite{Kavousi2019},
\cite{Wang2018}, and \cite{Choudhary2021} but are not shown in the
main manuscript of this work.\par

For each combination of potential/model, chemical system, and phase we
perform calculations for the structure minimization, equation of state
(EOS), phonons, and elastic constants. To determine the
finite-temperature phase transitions we utilize an isothermal-isobaric
ensemble via molecular dynamics simulations. The thermostating and
barostating is done with Nose-Hoover and Parrinello-Rahman dynamics
and the ramping down or up of temperature is adjusted in a dynamical
manner which entails adjusting the target thermostat temperature every
few timesteps based on the ramp-rate. The pressure is kept constant at
1 bar.

\begin{figure}[ht!]
	\script{VisualizedStructures.py}
	\begin{centering}
		\includegraphics[width=\linewidth]{figures/NiTi_VisualizedStructures.png}
		\caption{ Initial \ce{NiTi} structures (i.e., unrelaxed) used for
			(a) \textit{B2}(221), (b) \textit{B19}(11), (c)
			\textit{B19'}(51), and (d) \textit{BCO}(61) with spacegroup
			numbers are given in the parentheses. The green atoms are
			\ce{Ni} and \ce{Ti} as gray and unit cells are highlighted with
			parameters shown on axis. For all unit cells the $\alpha$ and
			$\gamma$ angles are 90.0$^\circ$. For PtTi, (a-b) structures are
			used with different unit cell parameters and basis coordinates
			and then optimized accordingly.  }
		\label{fig:niti_structures}
	\end{centering}
\end{figure}

Prior to any property calculations or dynamics the structures are
optimized using the FIRE\cite{Bitzek2006} local optimizer with the
forces converged to $2\cdot 10^{-3}$eV/\AA\@. We observed that other
local optimization routines find the same ground-state structures. To
ensure that the unit cell parameters were adjusted along with the
atomic coordinates and that the crystal symmetry is not broken, we
utlize a constraint and filter during the optimization process. \par

\subsubsection{Equation of State}
The EOS can be obtained by varying the cell volume and calculating the
total system energy without allowing for atomic relaxations. Starting
with the obtained ground-state structures the unit cell volume is
adjusted by applying an isotropic strain,

\begin{equation}
	\label{eq:isotropic_strain}
	T=\begin{bmatrix}
		\epsilon & 0        & 0        \\
		0        & \epsilon & 0        \\
		0        & 0        & \epsilon \\
	\end{bmatrix}
\end{equation},

to the cell and scaling the fractional atomic coordinates. Here
$\epsilon$ is the fraction strain applied to the unit cell. To ensure
that the strain is indeed isotropic we provide a quality check on the
symmetry of the distorted structure using the spacegroup package
\textit{spglib} \cite{Togo2018}. The isotropic strain is applied in
compression and tension, from $-11\%$ to $11\%$, which gave a good
range to characterize the EOS curves and obtain analytical fits. We
fit the EOS using a third order inverse polynomial,

\begin{equation}
	\label{eq:eos_fit}
	E(V) = c_0 + c_1 t + c_2 t^2 + c_3 t^3, \quad \text{where} \quad t = V^{-\frac{1}{3}}
\end{equation},

where the coefficients are used to determine equilibrium volume,
system energy, and bulk modulus \cite{Alchagirov2003} at 0K. In order
to compare respective EOS amongst the different phases, we normalize
the strained volume of each phase with respect to the obtained
minimized ground-state volume. \par

\subsubsection{Phonon band structure}

Shape memory alloys derive their unique displacive phase
transformations from the dynamic instability that occurs during the
application of temperature, strain, or stress. The origin of this
instability occurs due to softening or symmetry breaking of the phonon
modes in the phase. Therefore, improved understanding of the phonon
characteristics of the different phases that can occur for a given SMA
is particular important. One of the challenges is in performing such
calculations, which typically employ density functional theory (DFT)
in order to determine the atomic forces, which in turn are used to
construct the dynamical matrix. However, classical interatomic
potentials can alleviate the computational burden. \par

Phonon calculations are performed using finite-difference method via
small displacement (i.e., harmonic approximation) of atoms in a
reference cell \cite{Alfe2009}. In the small displacement method,
atoms in the primitive cell are slightly displaced to calculate forces
on other atoms, constructing the force constant matrix, which, when
used with the dynamical matrix, allows computing phonon frequencies at
any q-vector in the Brillouin zone. This requires typically requires
the use of a supercell of the original unit cell. In order to ensure
convergence of this method, one needs to sweep different supercell
sizes and atomic displacements. This process is done for each
interatomic potential and structure. The supercells used in this work
have a repeat in any one direction ranging from $5-8$ and depends on
the crystal system. The atomic displacement ranges from
$0.03-0.26$\AA\,and was found to strongly depend on interatomic
potential and crystal system. Details on the settings are provided in
the appendix~\ref{sec:phonon_calc_config}. \par

In order to sample high-symmetry points in the irreducible Brillouin
zone (IBZ) for specific structures, we sample from locations specified
in the Bilbao crystallographic \textit{KVEC} utility\cite{Aroyo2014},
although the ASE structures object does provide a standardized routine
for sampling the irreducible Brillouin zone for the specific
structures.\cite{Setyawan2010}. We predominately focus on
high-symmetry points in the IBZ that correspond to the corners and
along directions known to exhibit phonon instabilities. The IBZ and
points symmetry point locations can be seen in appendix
\ref{sec:appx_ibz}. The final phonon dispersion diagrams are plot
along these points. \par

To better understand dynamic instability in the various structures,
the same isotropic strain in eq.~\ref{eq:isotropic_strain} is used and
the respective phonon band structure are calculated. These are then
plotted to observed the onset or change in stability or instability of
phonon different branches. \par

\subsubsection{Elastic Constants}

Similar to the phonon calculations the elastic constants within the
harmonic approximation the elastic constants can be determined from
linear-response of strain-stress calculations. Therefore, the
calculation requires the use of crystal symmetry along with atomic
displacements to determine respective elastic constant components. We
use the \textit{Elastic}\cite{Jochym2018} python package which
provides routines for generating the proper atomic displacement matrix
given specific crystal space group. The elastic constants are
determined by solving a set of linear equations using singular value
decomposition and given calculated strains and stress. The
displacements occur with a min and max of 0.5\% and a total of 10
points are used. \par

\subsubsection{Finite-Temperature Phase Transitions}

\subsection{Physics Based Potentials}
\label{sec:physics_potentials}

The EAM and MEAM potentials were developed to go beyond pair
interatomic potentials to treat point and extended defects in metal
and alloys systems. The motivation was taken from the physical results
of density functional theory and tight-binding methods, namely to add
to the pair potentials a energy functional which depends on the local
electron density which in turn is affected by the atomic
environment. These type of interatomic potentials are broadly referred
to as EAM or MEAM, with the later included direction (i.e. angle)
dependent interactions. \par

The functional form using during fitting proceed depends on the
complexity of the material system but the total energy is typically
written as,

\begin{equation}
	\label{eq:eam}
	E = \sum_{i\neq j}^N V_{ij}(r_{ij}) + \sum_i^N F_i\left(\rho(r_i)\right)
\end{equation}

where $V_{ij}$ is the pair potential, $F_i$, the functional, and
$\rho$ the density function that depends on the environment around
$r_i$. The MEAM potential extends this by including three-body terms
in the $\rho$ to capture features such as covalent bonding. The EAM
and MEAM potentials have been particular successfully in predicting
structure and properties of metals and alloys, including defect
formation/energies, phase transitions, grain boundary
stability/dynamics. One particular advantage of both EAM and MEAM, is
their analytical or tabulated forms allow for quick numerical
evaluation, with the primary reason being that the energies and forces
on a atom are accurately determined within a relatively small
interaction volume (i.e., 3-4 \AA interaction cutoff).  \par

For application of EAM to shape memory alloys, and in particular NiTi,
\citet{Mutter2010} initially devised an Finnis-Sinclair style EAM
potential to capture the structural phase transitions between
\textit{B19'} and \textit{B2} phases. It was latter shown by
\cite{Zhong2011} that some adjustments to the EAM parameters were
needed to properly capture the finite-temperature displacement
dynamics for the martensitic transition \cite{Gur2017}. More recently
\citet{Ko2015} developed a second nearest neighbor(2NN) MEAM
parameterization of NiTi that demonstrated very good predictive
performance and improvements over earlier potentials.

\subsection{Graph Neural Network Potentials}

The use of ML potentials for SMA has been demonstrated by
\citet{Tang2022}. In their work, they utilized the
DeePMD-Kit\cite{Wang2018} neural network framework to parameterize a
descriptor matrix from atomic configurations that preserves
permutations and symmetries. The descriptors are then feed into a
multilayer perceptron to calculate energies and forces. The neural
network interatomic potential demonstrated good accuracy of
ground-state properties with respect to DFT results and showed
agreement with experiments regarding temperature, stress, and defect
induced martensitic transformation. \paragraphmark

The ground-state predictions for \textit{B2}, \textit{B19'},
\textit{B19}, and \textit{BCO}\footnote{Referred to as \textit{B33}
	prototype in \citet{Tang2022}} in \citet{Tang2022} showed good
agreement. However, we note that deviation from Ni-Ti 1:1
stoichiometry or addition of a third species was not investigated nor
the inclusion of additional alloying species was explored. \par

The pre-trained GNN interatomic potentials explored in this work have
been parameterized over a large fraction of the interatomic potential
($>$ 80 elements) and therefore are not restricted to specific
chemistries or stoichiometries. The training data also includes
various phases for a given chemical system which permits great
flexibility, although broad transferability still requires
verification.  This allows for investigation of both
non-stoichiometric and multi-species SMA which previous physics and ML
potentials were not developed for. We do note that fine-tuning, the
process of adjusting GNN weights and biases using additional training
data, can be performed with the GNNs studied in this work. However,
this is out of the scope for this manuscript, given resources and time
needed to generate such training data for SMAs. \par

The M3GNet is a GNN that utilizes a 3-body bond-update procedure to
more accurately capture bond-order environment changes. The M3GNet has
been implemented in the MatGL framework and we utilize the
\texttt{M3GNet-MP-2021.2.8-PES} pretrained model.

For CHGNet, we have used the \texttt{v0.3} series pretrained model.

% The MACE potential pretrained
% \texttt{2023-12-03-mace-128-L1_epoch-199.model} checkpoint file.

\section{Results}
\label{sec:results}

In fig.~\ref{fig:opt_density} the density of the structures for
different potentials is shown. We observe that in general the
densities and lattice parameters are within agreement with previously
reported values. The optimized lattice parameters as well as basis
atoms for all structures and models are provided in
appendix~\ref{sec:opt_structures}. \par

\begin{figure}[ht!]
	\label{fig:opt_density}
	\centering
	\includegraphics[options]{figures/NiTi_Equilibrium_Density.png}
	\caption{Equilibrium Density for \ce{NiTi} structures with different
		models.}
\end{figure}

To asses the relative phase stability for a given composition is to
compare the cohesive energy, which is defined as,

\begin{equation}
	\label{eq:cohesive_energy}
	E_{c} = \frac{E_{\text{bulk}} - \sum_{i}^{N} n_i E_{\text{atom}_i}}{N}
\end{equation}

With $E_{\text{bulk}}$ the total energy of the system,
$\sum_i^N n_i E_{\text{atom}_i}$ is the sum over isolated atom
energies for $n_i$ species, and $N$ the number of atoms in bulk
reference. Classical interatomic potentials are offset with respect to
the isolated atom energies, so these terms are zero and neglected,
giving $E_{c} = \frac{E_{\text{bulk}}}{N}$. In fig.~\ref{fig:ecoh} the
bar chart for different phases and interatomic potentials models are
show. We compare these to dervided experimental values for
\textit{B2}\cite{Vandermause2024} and the value from The Materials
Project (MP) entry \texttt{MP-571} which corresponds to the
\textit{B2} phase \cite{MP--571}. In the figure we see that the EAM
and MEAM potentials show significantly lower cohesive energy compared
to the GNN potentials and are in better agreement with the
experimental value. Previous studies using the EAM and MEAM potentials
have been shown to reproduce previous experimental and DFT data
\cite{Haskins2016,Ko2015} such as cohesive energy. The GNNs predict a
more strongly bound and stable \textit{B2} structure but all are
consistent with the entry for The MP and calculation details can be
found in ref.~\cite{MP--571}. This consistency is resonable
considering all pretrained GNN models used in this work have been
developed with MP data. \par

\begin{figure}[ht!]
	\script{PlotCohesiveEnergy.py}
	\begin{centering}
		\includegraphics[width=\linewidth]{figures/NiTi_CohesiveEnergyPlot.png}
		\caption{ Comparison of predicted \ce{NiTi} cohesive energy for
			respective ground-state structure and model. The experimental
			data point has been extracted from ref.~\cite{Vandermause2024}
			and the Materials Project entry for the \ce{NiTi} \textit{B2}
			structure \cite{MP--571} has also been included.  }
		\label{fig:ecoh_niti}
	\end{centering}
\end{figure}

For the \ce{PtTi} system we only consider the \textit{B2} and
\textit{B19'} structures. The MEAM potential shows a lower cohesive
energy than all three GNN. Interestingly the difference between the
MEAM and GNN is roughly the same order, $\approx 2.1$ eV, which maybe
appears to be related to the MP dataset. To our knowledge most the
pretrained GNN models are using the MP dataset which defaults to DFT
calculation results that employ the meta-GGA R2SCAN functional. This
could be the reason for the difference as most EAM and MEAM potentials
are either parameterized from experimental or DFT data using the PBE
functional\cite{Mutter2010,Ko2015}. \par

\begin{figure}[ht!]
	\script{PlotCohesiveEnergy.py}
	\begin{centering}
		\includegraphics[width=\linewidth]{figures/PtTi_CohesiveEnergyPlot.png}
		\caption{ Comparison of predicted \ce{PtTi} cohesive energy for
			respective ground-state structure and model.  }
		\label{fig:ecoh_ptti}
	\end{centering}
\end{figure}

\subsection{NiTi: Equation of States}
\label{subsec:nitieos}

The EOS curves for the \ce{NiTi} structures \textit{B2}, \textit{B19},
\textit{B19'}, \textit{BCO} for the different potentials studied in
this work are shown in fig.~\ref{fig:eos}. The two EAM potentials,
show that at 0 strain ($V/V_o = 1.0$) both the \textit{B19P} and
\textit{B33} phases are stable. I believe this is the first reporting
of the \textit{B33} phase for these two fittings of the EAM potential
for \ce{NiTi}. Its important to note that these potentials have been
developed to capture the displacive phase transformation between
\textit{B19P} to \textit{B2}. This is shown in fig.~\ref{fig:eos}(a,b)
with the stability indicated upon compressive isotropic strain. A
mininum appears around $\approx 0.9$. At tensile isotropic strains the
\textit{B19P} phase is stable up till around $1.18$ where the
\textit{Pbcm} phase shows stability. An interesting note is that these
two fittings of the EAM potential show a double well curve for the
\textit{B2} phase. \par

In the UIP-GNN potentials M3GNet, CHGNet, and MACE, the equation of
states shown in fig.~\ref{fig:eos}(c-e) look fairly different from
thos in fig.~\ref{fig:eos}(a,b). For the M3GNet, at no strain, the
\textit{B33} appears to be the predicted stable phase and the
\textit{B2} phase becomes stable around $0.9$ (see
fig.~\ref{fig:eos}(c). The stability for \textit{B19} and
\textit{B19P} exit, in trade-off, between approximately $1.025$ and
$1.1$ tensile isotropic strain. Interestingly beyond a strain of $1.1$
the \textit{Pbcm} is dominate. The double-well curve shown for the EAM
potentials is not present for the B2-phase. The CHGNet potential
results shown in fig.~\ref{fig:eos}(d) are similar to the M3GNet, but
the stability window of the \textit{B2} occurs at a lower compressive
isotropic strain. \par

The MACE potential shown in fig.~\ref{fig:eos}(e) has a very different
set of EOS curves compard to the M3GNet and CHGNet. Here many of the
phases exhibit a stablity window, except the \textit{B33} phase, which
was stable in M3GNet and CHGNet. Interestingly, the \text{B19} phase,
not the \text{B19P} phase is shown to be the stable zero strain
phase. One unique perspective of the the UIP-GNN models is that they
provide a different displacive phase transformation pathway then that
shown by the EAM potentials. However, this does not include the
meta-stable R-phase or other potential meta-stable phases which can
also offer such transformation mechanisms. \par

Finally, the ALIGNN potential EOS was generated, but a few issues
arose during the calculations. The most prominant was the difficulty
in convergence of the structural optimzation to the specified
tolerance (see \ref{sec:methods}. Secondly the calculated cohesive
energies differs from the UIP-GNN by a factor $\approx 2$ and by $1$
eV for the EAM potentials. The \textit{B19P} phase is the stable at
zero strain and for a considerable portion of the compressive, but all
of the tensile. At compressive strain beyond $\approx 0.94$ the
\textit{B2} phase is stable.

\begin{figure}[ht!]
	\script{PlotEOS.py}
	\begin{centering}
		\includegraphics[width=\linewidth]{figures/NiTi_EquationOfStates.png}
		\caption{ The equation of state curves for different \ce{NiTi}
			structures for (a) Mutter (EAM) (b) Zhong (EAM) (c) Ko
			(2NN-MEAM) (d) M3GNet (e) CHGNet (f) MACE interatomic
			potentials. The x-axis is the ratio of the isotropic strained
			unit cell volume, $V$, with respect to the ground-state unit
			cell volume, $V_o$.  }
		\label{fig:eos}
	\end{centering}
\end{figure}

\subsection{NiTi: Phonon Instability}
\label{subsec:nitiphonons}

\begin{figure}[!htp]
	\script{PlotPhonons.py}
	\begin{centering}
		\includegraphics[width=\linewidth]{figures/NiTi_B2_ModelsPhononBandstructure.png}
		\caption{ Ground-state, no isotropic strain, Phonon bandstructure
			of B2 phase for all models. The Mutter and Zhong EAM potentials
			differ in that Mutter predicts no dynamic instability of the B2
			phase at 0K.  }
		\label{fig:allmodels_b2}
	\end{centering}
\end{figure}

\begin{figure}[!htp]
	\script{PlotPhonons.py}
	\begin{centering}
		\includegraphics[width=\linewidth]{figures/NiTi_B19_ModelsPhononBandstructure.png}
		\caption{ Ground-state phonon bandstructure of B19 phase for all
			models.  }
		\label{fig:allmodels_B19P}
	\end{centering}
\end{figure}

\begin{figure}[!htp]
	\script{PlotPhonons.py}
	\begin{centering}
		\includegraphics[width=\linewidth]{figures/NiTi_B19P_ModelsPhononBandstructure.png}
		\caption{ Ground-state phonon bandstructure of
			\textit{B19}$^\prime$ phase for all models.  }
		\label{fig:allmodels_B19P}
	\end{centering}
\end{figure}

\begin{figure}[!htp]
	\script{PlotPhonons.py}
	\begin{centering}
		\includegraphics[width=\linewidth]{figures/NiTi_BCO_ModelsPhononBandstructure.png}
		\caption{ Ground-state phonon bandstructure of BCO phase for all
			models.  }
		\label{fig:allmodels_B19P}
	\end{centering}
\end{figure}

\subsubsection{Mode Grun\"{e}isen Parameter}

The B2 phase of NiTi exhibits notable phonon instabilities,
particularly at the \textit{M} point in the Brillouin zone. These
instabilities are characterized by imaginary modes, represented as
negative phonon frequencies, indicating a dynamical instability in
this crystal structure. The \textit{M} point, located at the edge of
the Brillouin zone, corresponds to a wave vector of
$\left[\frac{1}{2}, \frac{1}{2}, 0\right]$ in units of the reciprocal
lattice vectors. The phonon dispersion curves for \ce{NiTi} in the B2
phase reveal a softening of the acoustic modes at the \textit{M}
point, suggesting a propensity for a structural phase transition. \par

The mode Gruneisen parameter, $\gamma$, plays a pivotal role in
understanding the stabilization of the NiTi B2 phase under external
strain, particularly in the context of phonon instabilities at the M
point. The mode Gruneisen parameter is defined as,

\begin{equation}
	\label{eq:modegruneisen}
	\gamma = -\frac{V}{\omega}\left(\frac{\partial \omega}{\partial V}\right)
\end{equation},

where $V$ is the volume, usually taken at equilibrium, and $\omega$ is
the phonon frequency. For the \textit{B2} phase of \ce{NiTi}, the
negative phonon frequencies at the \textit{M} point indicate an
instability that can be modulated by applying strain. The strain
perturbs the lattice volume, thereby altering the phonon
frequencies. The effect of this strain can be quantitatively assessed
using the mode Gr\"{u}neisen parameter, which describes how the phonon
frequencies change with volume. A positive $\gamma$ suggests that the
phonon frequency increases as the volume decreases under strain,
potentially stabilizing the lattice. In the case of NiTi, applying
compressive strain may increase the frequencies of the soft modes at
the \textit{M} point, thus stabilizing the \textit{B2} phase and
delaying or suppressing the martensitic transformation. This interplay
between strain and phonon behavior, quantified by $\gamma$, is crucial
in tailoring the mechanical properties of shape memory and
superelastic materials.

In fig.~\ref{fig:modegruneisen} the Gr\"{u}neisen parameter for
\textit{M}-mode in \textit{B2} is shown for different interatomic
potentials for \ce{NiTi}. The instable modes (i.e., negative
eigenfrequencies) for the MEAM, M3GNet, and MACE potentials have
Gr\"{u}neisen parameter which are negative. The negative values
suggest that compressive isotropic strains will provide stabalizing
force against displacive phase transformation. At higher frequencies
the Gr\"{u}neisen parameter is positive, indicating this modes
experience a decrease in frequency with respect to volume
increase. The optical modes are less sensitive as indicated by smaller
values. \par

\begin{figure}[!htp]
	\script{GruneisenParameters.py} \centering
	\includegraphics[width=\columnwidth]{figures/Plot_NiTi_M_ModeGruneisen.png}
	\label{fig:modegruneisen}
	\caption{The sign-log of the \textit{M}-mode Gr\"{u}neisen
	parameter, $\gamma$, as a function of eigenfrequency for different
	interatomic potentials.}
\end{figure}

The Mutter EAM potential shows negative Gr\"{u}''neisen parameters for
all eigenfrequencies indicating compressive forces increase
frequencies, however, none of the \textit{M}-point modes have negative
eigenfrequencies. \par

In contrast the the MEAM, M3GNet, and MACE potentials, the Zhong EAM
and CHGNet potentials show positive Gr\"{u}niesen parameters for
negatives eignefrequencies, suggesting that isotropic tensile strain
acts to stabalize the mode. \par

To better coberate the results in fig.~\ref{fig:modegruneisen}, the
phonon band structure as a result of isoptroic strain applied in
compression and tension are show in fig

\begin{figure}[!htp]
	\script{PlotPhonons.py}
	\begin{centering}
		\includegraphics[width=0.5\linewidth]{figures/NiTi_Mutter_B2_StrainsPhononbandstructure.png} \vspace{1mm}
		\includegraphics[width=0.5\linewidth]{figures/NiTi_Zhong_B2_StrainsPhononBandstructure.png} \vspace{1mm}
		\includegraphics[width=0.5\linewidth]{figures/NiTi_Ko_B2_StrainsPhononBandstructure.png}
		\caption{ Phonon dispersion of B2 structure for the (top) EAM
			Mutter, (middle) EAM Zhong, and (bottom) 2NN-MEAM Ko potentials
			at different isotropic strains. The optical branches have been
			made more transparent to emphasis the behavior in the TA$_1$ and
			TA$_2$ branches. The color indicates direction from compressive
			to tensile strain. Note that the strain percent shown is with
			respect to isotropic changes (i.e. volume).  }
		\label{fig:mutter_zhong_ko_phonon_b2}
	\end{centering}
\end{figure}

% ADD Plot of zero strain EAM and first-pricinpals results

\begin{figure}[!htp]
	\script{PlotPhonons.py}
	\begin{centering}
		\includegraphics[width=0.5\linewidth]{figures/NiTi_M3GNet_B2_StrainsPhononBandstructure.png} \vspace{1mm}
		\includegraphics[width=0.5\linewidth]{figures/NiTi_CHGNet_B2_StrainsPhononBandstructure.png} \vspace{1mm}
		\includegraphics[width=0.5\linewidth]{figures/NiTi_MACE_B2_StrainsPhononBandstructure.png}
		\caption{ Phonon dispersion of B2 structure for the (top) M3GNet,
			(middle) CHGNet, and (bottom) MACE potentials for different
			isotropic strains. Similar transcparency is used as done in
			\ref{fig:mutter_zhong_phonon_b2}.  }
		\label{fig:gnn_phonon_b2}
	\end{centering}
\end{figure}

% \variable{output/Table_NiTi_M_ModeGruneisen.tex}

\subsubsection{Elastic Constants}

\variable{output/Table_NiTi_Elastic_Constants.tex}

\subsection{PtTi: Equation of States}
\label{sec:ptti_eos}

\section{Conclusion}
\label{sec:conclusion}
This word aimed to provide an assement of the of GNN univeral
interatomic potentials on the ability to capture the stable
ground-state phases of binary and ternary SMA. In addition, the
dynamic stability of these SMA is crucially evaluated to understand
the suitability in observing the onset of displacive phase
transformations via the softening of TA branch modes. In the case of
NiTi we find that all the GNN potentials predict either the Monoclinic
or BCO phases as stable over the B2 structure. However, not all agree
with the recent works by (cite) which have indicated that it is the
BCO phase which is the ground-state structure.

We observe that the M3GNet and CHGnet capture the M-point instability
of the B2 phase, however, no stablizing isotropic strain is
observed. With regard to the MACE, very good agrement is found with
the phonon bandstructure and that from ab-initio methods. The ALIGNN
potential was found not to be suitable of NiTi.

\bibliographystyle{elsarticle-num-names} \bibliography{references}

\section*{CR\lowercase{e}d\lowercase{i}T author statement}

S.B: Conceptualization, Formal Analysis, Investigation, Methodology,
Software, Visualisation, Writing---original draft.

\section*{Reproducibility and Data Availability}
This study was carried out using the reproducibility software
\href{https://github.com/showyourwork/showyourwork}{\showyourwork}\cite{Luger2021},
which leverages continuous integration to programmatically download
the data from \href{https://zenodo.org/}{zenodo.org}(if needed),
create the figures, and compile the manuscript. Each figure caption
contains two links: one to the dataset stored on zenodo used in the
corresponding figure, and the other to the script used to make the
figure (at the commit corresponding to the current build of the
manuscript). The git repository associated to this study is publicly
available at
\url{https://github.com/stefanbringuier/SMA_Phonons_GNNIP} and allows
anyone to re-build the entire manuscript.\par

\section*{Acknowledgements}

\newpage
\appendix
\section{Appendix}
\subsection{Workflow Reproducibility}
\label{sec:appx_reproduce}

TODO

\subsection{NiTi: Ground State Structures}
\label{sec:opt_structures}

\variable{output/Table_NiTi_Equilibrium_Structures.tex}

\subsection{Phonon Calculation Configurations}
\label{sec:phonon_calc_config}

The code used to generate the phonon configuration settings for each
potential mode and structure are shown the table below. Settings are
used for both NiTi and PtTi systems.

\variable{output/Phonons_ConfigSettingsCode.tex}

\subsection{Irreducible Brillouin Zones}
\label{sec:appx_ibz}

The irreducible Brillouin zones for the the different crystals
investigated in this study are provided below. The high-symmetry
points are indicated. Additionally the coordinates of the
high-symmetry points in the IBZ are shown in the tables below. In
Ref. xxx the specific implementation of the band path sampling and
definitions are provided.

\begin{figure}[!htp]
	\centering
	\begin{subfigure}[b]{0.45\linewidth}
		\includegraphics[width=\linewidth]{figures/B2_BrillouinZonePointsSampled.png}
		\caption{}
		\label{fig:B2}
	\end{subfigure}
	\hfill
	\begin{subfigure}[b]{0.45\linewidth}
		\includegraphics[width=\linewidth]{figures/B19_BrillouinZonePointsSampled.png}
		\caption{}
		\label{fig:B19}
	\end{subfigure}
	\vspace{1mm}
	\begin{subfigure}[b]{0.45\linewidth}
		\includegraphics[width=\linewidth]{figures/B19P_BrillouinZonePointsSampled.png}
		\caption{}
		\label{fig:B19P}
	\end{subfigure}
	\hfill
	\begin{subfigure}[b]{0.45\linewidth}
		\includegraphics[width=\linewidth]{figures/BCO_BrillouinZonePointsSampled.png}
		\caption{}
		\label{fig:BCO}
	\end{subfigure}
	\caption{The irreducible Brillouin zones for \textit{B2}(\#221),
		\textit{B19}(\#51), \textit{B19'}(\#11), and \textit{BCO}(\#63)
		crystal structures}
	\label{fig:ibz}
\end{figure}

\pagebreak

\variable{output/B2_SpecialSymmetryPointsBZ.tex}
\variable{output/B19_SpecialSymmetryPointsBZ.tex}
\variable{output/B19P_SpecialSymmetryPointsBZ.tex}
\variable{output/BCO_SpecialSymmetryPointsBZ.tex}

\end{document}
